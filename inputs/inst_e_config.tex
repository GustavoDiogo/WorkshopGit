\section{Instalar e Configurar}

\begin{slide}[method=direct]{Instando o GIT no Linux}
	\begin{itemize}
	\item{Para instalar o Git no Ubuntu, ou em outra distribuição baseada em Debian, execute no terminal o seguinte comandos:}
		\begin{lstlisting}[style=Bash]
$ sudo apt-get install git
		\end{lstlisting}
	
		E para quem utiliza Fedora, utilize:
	
		\begin{lstlisting}[style=Bash]
$ sudo yum install git
		\end{lstlisting}
	\end{itemize}
\end{slide}

\begin{slide}[method=direct]{Configurando o GIT}
	\begin{itemize}
	\item{Sempre que instalamos o GIT, é necessario informar para o GIT, quem somos, para isto usamos as seguintes linhas de comando:
		\begin{lstlisting}[style=Bash]
$ git config --global user.name "NOME"
$ git config --global user.email E@MAIL
		\end{lstlisting}}
	\end{itemize}
\end{slide}
