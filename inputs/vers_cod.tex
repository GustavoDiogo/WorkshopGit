\section{Versionando seu código}

\begin{slide}[method=direct]{Criando um repositório}
	\begin{itemize}
		\item{Para criar e inicializar um repositório, navegue até o diretorio onde vai ficar os arquivos do seu projeto e digite o comando:
			\begin{lstlisting}[style=Bash]
$ git init
			\end{lstlisting}
		Após executar o comando, deve aparecer uma mensagem semelhante a esta:
		\begin{lstlisting}
Initialized empty Git repository in /*/.git/
		\end{lstlisting}
		}
		
		\item{Com isso já temos um repositorio, vazio, do Git inicializado, atente que o sistema criou automaticamente uma pasta chamada .git ( em modo oculto )}
	\end{itemize}
\end{slide}

\begin{slide}[method=direct]{Status do repositório}
	\begin{itemize}
		\item{Quando criamos ou adicionamos um arquivo na pasta onde foi inicializado o projeto o GIT não inclui automaticamente o novo arquivo, para isso é necessario ``rastrear'' o arquivo e adicionar ele, para isto, utilizamos o comando:
			\begin{lstlisting}[style=Bash]
$ git status
			\end{lstlisting}
			
			O resultado do comando será algo semelhante a:
			\begin{lstlisting}[basicstyle=\tiny]
$ git status
On branch master

Initial commit

Untracked files:
  (use "git add <file>..." to include in what will be committed)

        Makefile
        doc/
        fonte1.c
        fonte2.asm
        
        (...)
\end{lstlisting}
			}
	\end{itemize}
\end{slide}

\begin{slide}[method=direct]{Rastreando arquivos}
	\begin{itemize}
	\item{Para que um arquivo passe a ser rastreado pelo Git, devemos executar o seguinte comando:
		\begin{lstlisting}[style=Bash]
$ git add <nome do arquivo>
		\end{lstlisting}
	       	
	        Exemplo:
	        \begin{lstlisting}[style=Bash]
$ git add fonte2.asm
	        \end{lstlisting}
	        }
	\item{Após eecutar o comando \textit{git status} iremos ver algo semelhante com}
	\begin{lstlisting}[basicstyle=\tiny]
On branch master
Initial commit

Changes to be committed:
  (use "git rm --cached <file>..." to unstage)

        new file:   fonte2.asm

Untracked files:
  (use "git add <file>..." to include in what will be committed)

        Makefile
        (...)
	\end{lstlisting}
	\end{itemize}
\end{slide}

\begin{slide}[method=direct]{Gravando o arquivo no repo.}
	\begin{itemize}
	\item{Para gravarmos as mudanças no repositório, devemos executar o comando:
	        \begin{lstlisting}[style=Bash]
$ git commit -m ``Meu primeiro commit!!!''
	        \end{lstlisting}
	        Observe que o comando \textit{git commit} foi executado junto com o parâmetro \textit{-m}, que é utilizado para definir uma mensagem para o commit que você está submetendo ao servidor.
	        \begin{itemize}
	        \item{\textbf{A mensagem do commit deve ser muito clara, ao descrever quais são as modificações que você está enviando para o servidor!}}
	        \end{itemize}
	        Após executar o git commit, você verá:
	        \begin{lstlisting}[basicstyle=\tiny]
git commit -m "Meu primeiro commit!!!"
[master (root-commit) ff932de] Meu primeiro commit!!!
 1 file changed, 0 insertions(+), 0 deletions(-)
 create mode 100644 fonte2.asm
	         \end{lstlisting}
	        }
	\end{itemize}
\end{slide}

\begin{slide}[method=direct]{Alterando arquivos}
	\begin{itemize}
	\item{Se editarmos um arquivo já versionado pelo git, quando executamos o comando \textit{git status}, ele ira nos retornar o seguinte resultado:}
	\begin{lstlisting}[basicstyle=\tiny]
$ git status
On branch master
Changes not staged for commit:
  (use "git add <file>..." to update what will be committed)
  (use "git checkout -- <file>..." to discard changes in working directory)

        modified:   fonte2.asm
	\end{lstlisting}
	\end{itemize}
\end{slide}

\begin{slide}[method=direct]{Verificando as alterações}
	\begin{itemize}
	\item{Podemos verificar o histórico das alterações gravadas no repositório com a seguinte linha de comando:
	\begin{lstlisting}[style=Bash]
$ git log
	\end{lstlisting}
	O resultado do comando vai ser parecido com:
	\begin{lstlisting}[basicstyle=\tiny]
$ git log
commit 182279fb11c39d0830825fa0c75366c4a9905c1d
Author: Alexandre <alebencz@gmail.com>
Date:   Thu Mar 10 13:34:30 2016 -0300

    Corre<C3><A7><C3><A3>o no fonte2.asm e adicionado
    objeto para ser compilado pelo make

commit ff932ded962e4d2029eba37a879d0886036ea600
Author: Alexandre <alebencz@gmail.com>
Date:   Thu Mar 10 13:11:13 2016 -0300

    Meu primeiro commit!!!
          \end{lstlisting}
	}
	\end{itemize}
\end{slide}