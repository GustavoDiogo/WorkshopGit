\section{Se organizando com as branches}

\begin{slide}[method=direct]{Trabalhando em paralelo}
	\begin{itemize}
	\item{Muitos sistemas de controle de versão permite trabalho em paralelo através de \textbf{branches}}
	\item{Mas o que é uma Branch ?}
	\begin{itemize}
		\item{Uma \textbf{branch} é uma linha independente de desenvolvimento em que podemos enviar novas versões do código sem alterar as outras branches}
	\end{itemize}
	\end{itemize}
\end{slide}

\begin{slide}[method=direct]{Criando uma branch}
	\begin{itemize}
	\item{Para criar uma branch, basta executar o seguinte comando:
		\begin{lstlisting}[style=Bash]
$ git branch <nome da nova branch>
		\end{lstlisting}
		A execução deste comando não retorna nenhuma resposta, então, para listartmos as branchs existentes, utilizamos o comando:
		\begin{lstlisting}[style=Bash]
$ git branch
		\end{lstlisting}
		O resultado deste comando vai retornar a lista de todas as branchs existentes par ao projeto
		\begin{lstlisting}[style=Bash]
$ git branch
  current
* master
                   \end{lstlisting}
	         }
	\end{itemize}
\end{slide}

\begin{slide}[method=direct]{Trocando de branch}
	\begin{itemize}
	\item{Para trocarmos de uma branch para outra, executamos o seguinte comando:
	\begin{lstlisting}[style=Bash]
$ git checkout <nome da nova branch>
	\end{lstlisting}
	Quando executado este comando, deve aparecer como resposta algo como:
	\begin{lstlisting}[style=Bash]
Switched to branch 'current'
	\end{lstlisting}
	}
	\end{itemize}
\end{slide}

\begin{slide}[method=direct]{Criar e trocar para uma nova Branch}
	\begin{itemize}
	\item{Para uma questão de facilidade, o GIT fornece uma opção para criar e autoamticamente trocar de branch, quando executado o comando de checkout, o parametro \textit{-b} deve ser passado antes do nome da nova branch, executando o mesmo comando para trocar de uma branch.
	\begin{lstlisting}[style=Bash]
$ git checkout -b <nome da nova branch>
	\end{lstlisting}
	Após executar este comando, a saida será:
	\begin{lstlisting}[style=Bash]
$ git checkout -b v1.0
M       fonte2.asm
Switched to a new branch 'v1.0'
	\end{lstlisting}
	}
	\end{itemize}
\end{slide}

\begin{slide}[method=direct]{Commitando para uma branch}
	\begin{itemize}
	\item{Para commitar para uma nova branch, no momento de executar o comando \textit{push}, deve ser informado o nome da branch para qual o commit vai ser enviado:
	\begin{lstlisting}[style=Bash]
$ git push origin v1.0
	\end{lstlisting}
	Após executado o comando, o resultado será algo como:
	\begin{lstlisting}[style=Bash,basicstyle=\tiny]
$ git push origin v1.0
Counting objects: 4, done.
Delta compression using up to 2 threads.
Compressing objects: 100% (2/2), done.
Writing objects: 100% (4/4), 385 bytes | 0 bytes/s, done.
Total 4 (delta 0), reused 0 (delta 0)
To https://github.com/bencz/RepositorioDeEstudo.git
 * [new branch]      v1.0 -> v1.0
 	\end{lstlisting}
	}
	\end{itemize}
\end{slide}

\begin{slide}[method=direct]{Fazendo merge}
	\begin{itemize}
	\item{Para juntarmos as alterações feita em outras branchs com a branch \textit{master}, podemos utilizar o seguinte comando: ( Lembrando que, você deve estar na branch \textit{master} para fazer o merge com os dados da branch v1.0
	\begin{lstlisting}[style=Bash,basicstyle=\tiny]
$ git merge v1.0 -m ``Fazendo merge com a branch v1.0''
	\end{lstlisting}
	O resultado da execução deste comando será:
	\begin{lstlisting}[style=Bash,basicstyle=\tiny]
$ git merge v1.0 -m "Fazendo merge com a branch v1.0"
Updating 182279f..456bc06
Fast-forward (no commit created; -m option ignored)
 fonte1.c   | 3 +++
 fonte2.asm | 2 ++
 2 files changed, 5 insertions(+)
 create mode 100644 fonte1.c
 	\end{lstlisting}
 	Feito isso, basta executar o comando \textit{git push origin master} e os dados vão estar mesclados
	}
	\end{itemize}
\end{slide}

\begin{slide}[method=direct]{Deletando uma branch}
	\begin{itemize}
	\item{Para deletarmos uma branch, devemos utilizar a opção \textit{-d} junto ao comando \textit{git branch}.
	\begin{lstlisting}[style=Bash]
$ git branch -d v1.0
	\end{lstlisting}
	O resultado da execução deste comando será:
	\begin{lstlisting}[style=Bash]
$ git branch -d v1.0
Deleted branch v1.0 (was 456bc06).
	\end{lstlisting}
	}
	\end{itemize}
\end{slide}